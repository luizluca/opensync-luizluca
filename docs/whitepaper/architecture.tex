\chapter{Framework Architecture}

\section{IPC}
Part of the OpenSync framework is an Interprocess Communication system to
communicate with the Plugins. To avoid confusion between the different process
types of Plugins, and to make the IPC independent of Plugins, the neutral terms
">Client"< and >"Client Proxy<" got introduced. The IPC is allows full duplex 
(bidirectional) communication between Client and Proxy. Independent of the 
process type of the Client. The frameworks allows to communicate between three 
different process types of Clients:
\begin{itemize}
\item Threaded via pipes (two pipes for full duplex/bidirectional ipc. See
pipe(7))
\item Forked via pipes (two pipes for full duplex/bidirectional ipc. See pipe(7))
\item External process via named pipes (FIFO)
\end{itemize}
This IPC also to send specific type of messages between the Client and the
Proxy. The Proxy takes care about timeout handling, if a client doesn't
response in time. This is only supported for messages which have a callback
function assigned. Such callbacks (command handler) get executed as soon if the
message reply by the Client is retrieved. On a timeout or error the callback
function got called with the custom error or timeout message.

\subsection{Client}
Clients handle all requests from the Proxy. Each request from the proxy gets
replied. Threaded and Forked clients get started by the Proxy. Most common use of
the Client is to call Plugin specific functions, requested with message send by
the Proxy. The client replies the result of the function calls with replying to
the messaged which requested the event.

\subsection{Proxy}
The Proxy handles all request to and from the Client. For each Client there is
one Proxy. The proxy got mainly used by the OpenSync framework engines to
communicate with the Plugin processes.

\subsection{Messaging Queues}
The communication between Client and Proxy is based on message queues. There are
two types of such Queues: Sender- and Receiver-Queue. The Queue object supports
(anonymous) pipes as well as named pipes (FIFO).\\
\\
Note: When using Queues along forked processes the read and write ends have to
be disconnected/closed for proper communication. See pipe(7).\\
\\
There is a specified set of Message Commands for simple communication between
Client and Proxy. For synchronized communication between Client and Proxy the
Client simply replies the Message Command from the Proxy, when Client executed
the requested command. If Client failed while executing the command, it has to
reply with an error reply. Synchronize communication requires that the caller
assign a message handler to the Message object, which got called when the
message got replied. This message handler should check carefully the received
replied message type, and handle provide proper error handling.\\
\\
For synchronized communication it's recommended to send message with a timeout
via the message queue. To avoid deadlocks when the counterpart fails to reply.
No timeout handling is needed for asynchronized communication without command
handler, since no reply is expected.\\
\\
Additional content of the message have to be marshaled/demarshaled, to make the
content usable even it's running in a different address space. (rough rule of 
thumb: everything which includes a pointers/memory addresses)

\section{Engines}
So called engines are the core of the entire framework and control the entire
sequence of every single step which is done during a synchronization. When the
engine got initialized it checks if a previous synchronization was unclean
(failed) and a slow sync is required. If the previous synchronization was
unclean or not could be only determined by engine itself if an group environment
is used. Beside the group environment it also initializes the plugin, format.
The engine isn't restricted to get called with a group environment, the group 
environment is only of optional use.\\
\\
If the group environment is used, the engine locks the group when the engine
gets initialized. If the group detects an unassigned lock the engine will
request while synchronization a slow sync.\\
\\
To determine a previous unclean synchronization without a group environment , 
it's up to OpenSync framework using program to provide facilities to store such 
information. If so, the engine MUST be set to trigger a slow sync, after it got 
initialized. When the engine got finalized or synchronized successful the slow 
sync engine status got reseted.\\
\\
With the engine initialization also the members of the group got initialized.
The member initialization creates for corresponding member plugin a Client Proxy
which got initialized and spawns the plugin. For more details check the IPC
section about the Proxy.\\
\\
After all members got initialized, the >"Object Engines<" for each enabled
Object Type got initialized. If no Object Type is enabled or discovered the
engine initialization aborts with an error. The engine needs at least one Object
Type to synchronize. Often this error appears if not all or none members got
discovered. See discover section of "Merging capabilities" for more details.

\subsection{Object Engine}
For each Object Type, which have to be synchronized, one Object Engine gets 
started by the (main) engine. If the engine detected that previous 
synchronization was unclean, each Object Engine get the slow sync flag set to 
perform a slow sync on the next sync. If the Object Engine got finalized or 
synchronized (successful or unsuccessful) the slow sync flag get reseted. If 
the synchronization failed, it's most likely that the (main) engine will set 
the slow sync flag again on the next synchronization.\\
\\
The Object Engines allows to perform actions only for a certain Object Type. For
example to request a slow sync only for specific Object Engines (this means also
for specific Object Types).\\
\\
The initialization of an Object Engine creates for each Client Proxy (read 
Members' Plugin) a Sink Engine, which maps each Client Proxy with an Object
Type. (More about the Sink Engine in the next section.)\\
\\
If a slow sync got set for the Object Engine, then all >"Mappings<" with the
Object Type of the Object Engine got deleted from the >"Mapping Table<". This is
done to avoid loss of the data inconsistence while performing a slow sync.\\
\\
If no slow sync got set for this Object Engine the >"Mapping Table<" got loaded
and creates for each Mapping a >"Mapping Engine<".

%TODO: Injecting ignored entries in mapping engine

%TODO


\subsection{Mapping Engine}
\subsection{Mapping Entry Engine}


\subsection{Sink Engine}
The Sink Engine currently doesn't consists of any logic. The only relation to
Engines of the Sink Engine object is, it maps all the already mentioned engines
together.\\
\\
The Sink Engine maps the Client Proxy and Object Engine together. With this 
combination the Sink Engine is able to handle Object Type specific (aka. 
ObjTypeSink) tasks. Additionally it contains of a list of Mapping Engines, 
which handle Object Type and Client related Mappings only.

\section{Merging Different Capabilities}
\subsection{Capabilities}
\subsection{Archive}
\subsection{Merger \& Demerger}
\section{Helpers}
\subsection{Anchor Table}
\subsection{Hash Table}
\subsection{Time Helper}
\section{Filter}
\subsection{Custom Fiter}
