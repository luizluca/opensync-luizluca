\chapter{Introduction}
\label{chap:intro}
OpenSync is intended to be an universal Synchronization Framework, not limited to
a platform or desktop nor to PIM!

\section{History}
MultiSync, started by Bo Lincoln, was intended to synchronize various mobiles,
PDAs and PIM applications. MultiSync was a GTK application and provided
support for different PIM applications and devices by providing plugins.
The primary focus of MultiSync was to synchronize just PIM data (contacts, 
appointments, notes). Unfortunately this approach was fixed to a single UI and 
limited to PIM synchronization. Development branch MultiSync 0.9x was 
intended fix those needs.\\
\\
Early in 2005, Armin Bauer started refactoring the MultiSync project and
separated the UI code from MultiSync and the Plugin interface and
Synchronization logic. The result of his work is the most powerful and
flexible Synchronization Framework, known today as OpenSync.
\section{Big Picture}
OpenSync is intended to provide a low-level synchronization API for all kinds of 
data.
\\
\begin{figure}
 \centering
 % \includegraphics[bb=0 0 222 140,scale=0.5]{bigpicture}
 % bigpicture.png: 222x140 pixel, 72dpi, 7.83x4.94 cm, bb=0 0 222 140
 \caption{OpenSync}
 \label{fig:bigpicture}
\end{figure}


\section{Goals}
The goals of the OpenSync project are:

\begin{itemize}
 \item Reusabilty. OpenSync should be usable for all kinds of synchronization 
 applications
 \item Speed. Synchronization should be efficent and as fast as possible to give the
 user the best experience.
 \item Flexibility. Nobody can predict what will come next to synchronize after the
 G- \& iPhone, so OpenSync has to be designed and built as flexible and modular 
 as possible.
 \item Integrity. Data MUST never be lost or malformed, no matter what happens. 
 Data loss is just a no go!
 \item Portability. OpenSync should run on as many platforms as possible as 
 possible.
 \item Fun. The experience of using and developing OpenSync MUST be fun! You 
 should not work on OpenSync if you're unhappy :)
\end{itemize}
Not all of the mentioned goals are achieved yet, but OpenSync tries to get 
close as possible over time.

\section{Who should read this document?}
This Document is intended for everyone who is interested in a Open Source 
Synchronization solution. This is not a HowTo, Tutorial or Guide on how to use 
OpenSync. It is a detailed introduction of the OpenSync framework. No (high) 
programming skills/experince are required for this document.

\section{How should this document be read?}
Some chapter are very detailed about internal technial stuff, which is mainly 
intended for developers or people who are interested in how OpenSync works. Those
chapters could be safely skipped if someone isn't interested in touching the
internals of the OpenSync Framework. Such sections are marked with a little hint.

\section{Is this the most recent version of this paper?}
Maybe not, OpenSync is in a very early stage of development and lots of stuff
might changed within the last weeks. If in doubt, you might compare your version 
with the one from:\\
\\
\hyperref[http://www.opensync.org/download/LATEST-WHITEPAPER]{http://www.opensync.org/download/LATEST-WHITEPAPER}\\
\\
This paper was created on: \today
